% !TeX root = ../ejemplo.tex

\section{Reconocimiento facial}

El reconocimiento facial es una técnica que permite a las computadoras predecir la identidad de una persona desde una imagen [1].

Tiene múltiples aplicaciones en la industria como el desarrollo de sistemas de asistencia, en la salud, sistemas de seguridad, etc [2]. 

\subsection{Fundamentos del reconocimiento facial}

A menudo, los términos verificación de rostros e identificación de rostros se usan de forma indistinta, sin embargo, aunque ambos comparten el mismo dominio del problema, lo abordan de forma distinta [3].

Para entender mejor la diferencia entre ambas tareas y su papel en el desarrollo de un sistema de reconocimiento facial, es que se presentan las técnicas más comunes que lo conforman [2].

\subsubsection{Detección de rostros}

En este paso se incluyen algunas estrategias de preprocesamiento que permiten obtener imágenes de buena calidad para los métodos de extracción de características.
Aquí se localiza la región del rostro en una imagen o video y, dependiendo del contexto es posible aplicar técnicas de alineación o cortado de la imagen para aislar los elementos del fondo. Algoritmos como \textit{Viola-Jones} y modelos basados en \textit{Deep Learning}, como \textit{Multi-task Cascaded Convolutional Networks} (MTCNN), son ampliamente utilizados para esta tarea [3]. 

\subsubsection{Extracción de características}

Una vez que se tiene localizado el rostro se procede a extraer las características de este. En sistemas tradicionales, solían utilizarse descriptores como Histogramas de Gradientes Orientados (\textit{HOG}) o Escalas de Características Invariantes (\textit{SIFT}). Actualmente, los modelos basados en las redes neuronales profundas, especialmente, las redes neuronales convencionales (CNN), como \textit{FaceNet}, generan representaciones compactas conocidas como embeddings, que capturan la información más relevante del rostro [1].

\subsubsection{Comparación y verificación}

Los embeddings faciales obtenidos se comparan con una base de datos utilizando medidas de similitud como la distancia coseno o euclidiana. Este paso permite determinar si dos rostros corresponden a la misma persona (verificación) o identificar a una persona entre múltiples registros (identificación) [1].


\subsection{Identificación de rostros}

Como ya se mencionó, la identificación de rostros y la verificación de rostros son dos tareas distintas involucradas al momento de implementar un sistema de reconocimiento facial [1].

La identificación facial se refiere a la tarea de identificar la identidad de una persona dada una imagen.La imagen se ingresa por un extractor de características para obtener una representación $f$. Luego, esta representación, entra a una red de clasificación para asignar la identidad asociada a la imagen de entrada [2].

La identificación de rostros es una tarea de clasificación, por lo que los sistemas de reconocimiento facial basados en esta tarea son entrenados usando la función \textit{softmax} y la pérdida de entropía cruzada [4].

\subsection{Verificación de rostros}

Contrario a la identificación de rostros, la verificación de rostros busca verificar que dos imágenes pertenecen a la misma entidad [1]. Por lo tanto, en un sistema de reconocimiento facial basado en la verificación de rostros recibe como entrada una imagen $x$ y el identificador del rostro a ser identificado y la salida es una decisión binaria sobre si la imagen $x$ pertenece a la persona con el identificador dado [2].

Este enfoque requiere de un proceso de extracción de características de los rostros de las personas que queremos que el sistema verifique y, por lo tanto, almacenar dichas representaciones en una base de datos [2].

\subsection{Identificación mediante verificación}
En los sistemas tradicionales de identificación, se entrena un clasificador con $K$ clases para asignar a cada rostro de entrada una identidad específica de entre $K$ personas en la base de datos. Sin embargo, este enfoque presenta limitaciones de escalabilidad, ya que al agregar una nueva persona, es necesario reentrenar completamente el sistema con $K+1$ clases. Esto ocurre porque el clasificador inicial tiene $K$ neuronas, pero para incluir una nueva identidad, se requiere un clasificador con $K+1$ neuronas [2].

\vspace{\baselineskip}

Para abordar este problema, un enfoque más eficiente y escalable es utilizar la comparación basada en similitud, propia de los algoritmos de verificación. En este caso, dado un rostro de entrada $x$, se ejecuta el algoritmo de verificación $K$ veces, una para cada rostro almacenado en la base de datos. La identidad del rostro de entrada se asigna a la correspondiente al ID para el cual el algoritmo de verificación indica una coincidencia [2].

Por lo tanto, la ventaja de este tipo de enfoque híbrido radica en el hecho de que es posible agregar nuevas entidades al sistema sin la necesidad de re-entrenar los modelos de deep learning utilizados para la generación de los vectores de características de los rostros, más aún si se combina con algún tipo de mecanismo de almacenamiento de estos como las bases de datos vectoriales para acelerar el proceso de inferencia [1].





