% !TeX root = ../ejemplo.tex

\section{Framework}
\subsection{Jetpack compose}
Para la implementación de la aplicación móvil que forma parte de nuestro sistema de identificación y control de acceso, hemos decidido utilizar Jetpack Compose. La elección de la tecnología adecuada es importante para ofrecer una mejor experiencia de usuario y cumplir nuestros objetivos de diseño y funcional. 

\subsection*{¿Por que Jetpack compose?}
Jetpack compose es un framework (estructura o marco de trabajo que, bajo parámetros estandarizados, ejecutan tareas específicas en el desarrollo de un software) con la particularidad de ejecutar prácticas modernas en los desarrolladores de software a partir de la reutilización de componentes, así como también contando con la oportunidad de crear animaciones y temas oscuros. En este sentido, Jetpack Compose es el conjunto de herramientas ofrecidas por Android para el desarrollo de aplicaciones con un objetivo específico: simplificar y optimizar los códigos en la IU nativas \cite{cita30}. 


\subsection*{Ventajas}

\begin{itemize}
	\item \textbf{Menos código:} Simplifica el proceso de desarrollo haciendo menos código, todo se basa en funciones de modo que el código será simple y fácil de mantener.
	\item \textbf{Intuitiva:} Tan solo describe tu IU con un enfoque declarativo haciendo “qué hay que hacer” en vez de “cómo se debe hacer”.
	\item \textbf{Potente:} Tiene integrado Material Design con el cual puede crear apps atractivas al usuario con animaciones y mucho más.
	\item \textbf{Acelera el desarrollo:} Es compatible con proyectos existentes, puedes empezar a integrarlo por partes cuando quieras y donde quieras.
	\item \textbf{Kotlin:} Está escrito 100\% en Kotlin, lo cual nos permitirá usar sus herramientas potentes y API’s intuitivas.
\end{itemize}