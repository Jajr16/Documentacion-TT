% \IUref{IUAdmPS}{Administrar Planta de Selección}
% \IUref{IUModPS}{Modificar Planta de Selección}
% \IUref{IUEliPS}{Eliminar Planta de Selección}

%-------------------------------------- COMIENZA descripción del caso de uso.

%\begin{UseCase}[archivo de imágen]{UCX}{Nombre del Caso de uso}{
%--------------------------------------
\begin{UseCase}{CU17}{Consultar periodos de ETS inscritos del alumno}{
		Este caso de uso permite al alumno consultar los periodos de ETS. 
	}
	\UCitem{Versión}{\color{Gray}1.0}
	\UCitem{Autor}{\color{Gray}De la cruz De la cruz Alejandra}
	\UCitem{Supervisa}{\color{Gray}Ulises Velez Saldaña}
	\UCitem{Actor}{\hyperlink{Alumno}{Alumno}}
	\UCitem{Propósito}{Permitir al alumno consultar los periodos de ETS.}
	\UCitem{Entradas}{Ninguna}
	\UCitem{Origen}{Pantalla táctil}
	\UCitem{Salidas}{Lista de periodos de ETS.}
	\UCitem{Destino}{\IUref{IU14}{Pantalla Periodo de ETS alumno}}
	\UCitem{Precondiciones}{El alumno debe estar autenticado en el sistema.}
	\UCitem{Postcondiciones}{El alumno ha consultado los periodos de ETS.}
	\UCitem{Errores}{
			E1: El sistema no puede recuperar la información de los periodos.
			
			E2:  No hay periodos de ETS. 
	}
	\UCitem{Tipo}{Se extiende del CU16 Iniciar sesión de alumnos}
	\UCitem{Observaciones}{Ninguna}
\end{UseCase}
%--------------------------------------
\begin{UCtrayectoria}
	\UCpaso[\UCactor] Accede a la \IUref{IUE03}{Pantalla Menú del alumno} después de haber iniciado sesión.
	\UCpaso[\UCactor] Selecciona la opción \IUbutton{Consultar periodo de ETS}.
	\UCpaso Verifica que el alumno cuente con periodos de ETS. \Trayref{A}.
	\UCpaso Busca en la base de datos los periodos de ETS asignados al alumno.
	\UCpaso Despliega la lista de periodos de ETS asignados al alumno en la \IUref{IU14}{Pantalla Periodo de ETS alumno}.
\end{UCtrayectoria}
%--------------------------------------        
\begin{UCtrayectoriaA}{A}{No hay periodos}
	\UCpaso Verifica y no encuentra registros de periodos de ETS.
	\UCpaso Muestra un mensaje: {\bf MSG10-}{``No hay periodos de ETS''}
	\UCpaso[\UCactor] Presiona el botón \IUbutton{Regresar} para volver a la pantalla anterior.
	\UCpaso Fin de la trayectoria alternativa.
\end{UCtrayectoriaA}
%--------------------------------------        
\begin{UCtrayectoriaA}{B}{Error en la conexión con la base de datos}
	\UCpaso Muestra un mensaje de error: {\bf MSG11-}{``Error al consultar la base de datos. Intente nuevamente más tarde.''}
	\UCpaso[\UCactor] Presiona el botón \IUbutton{Aceptar} para cerrar el mensaje.
	\UCpaso[\UCactor] Puede intentar la consulta nuevamente o presionar el botón \IUbutton{Regresar} para volver a la pantalla anterior.
	\UCpaso Fin de la trayectoria alternativa.
\end{UCtrayectoriaA}
%-------------------------------------- TERMINA descripción del caso de uso.