% !TeX root = ../ejemplo.tex


\subsection{XML}
Para este proyecto utilizaremos XML debido a la múltiples ventajas que ofrece en el desarrollo de interfaces de usuario en aplicaciones web y Android\cite{CitaA03} .  

\subsection*{¿Por que XML?}

XML son siglas de \textit{Extensible Markup Language}, es un lenguaje de marcado que proporciona reglas para definir cualquier dato.

Por ejemplo, imaginemos un documento de texto con comentarios. Los comentarios pueden ofrecer sugerencias como las siguientes:
\begin{itemize}
	\item Ponga el título en negrita.
	\item Esta oración es un encabezado.
	\item Esta palabra es del autor.
\end{itemize}

Estos comentarios mejoran la usabilidad del documento sin repercutir en su contenido. Del mismo modo, XML utiliza símbolos de marcado para proporcionar más información sobre los datos.

\subsection*{Etiquetas XML}

Los s\'{\i}mbolos de marcado, denominados \textbf{etiquetas} en XML, se utilizan para definir los datos. Por ejemplo, para representar los datos de una librer\'{\i}a, se pueden crear etiquetas como:

\texttt{\textless libro\textgreater}, \texttt{\textless t\'{\i}tulo\textgreater} y \texttt{\textless autor\textgreater}

El documento XML de un solo libro tendr\'{\i}a el siguiente contenido:

\begin{lstlisting}
	<libro>
	<titulo>Introduccion a Amazon Web Services</titulo>
	<autor>Mark Wilkins</autor>
	</libro>
\end{lstlisting}


Las etiquetas ofrecen una sofisticada codificaci\'{o}n de datos para integrar los flujos de informaci\'{o}n en diferentes sistemas \cite{CitaA04}.

\subsection*{Ventajas de XML}

\begin{itemize}
	\item \textbf{Flexibilidad:} El formato XML es un lenguaje de marcas que se puede personalizar para diferentes prop\'{o}sitos.
	\item \textbf{Interoperabilidad:} El formato XML es compatible con una amplia gama de sistemas y aplicaciones, lo que significa que los datos se pueden intercambiar f\'{a}cilmente entre diferentes sistemas.
	\item \textbf{Legibilidad:} El formato XML es f\'{a}cil de leer y entender, lo que facilita la creaci\'{o}n y el mantenimiento de archivos XML.
	\item \textbf{Reutilizaci\'{o}n:} Los elementos y atributos de un archivo XML se pueden reutilizar en diferentes partes del archivo, lo que ahorra tiempo y reduce errores.
\end{itemize}