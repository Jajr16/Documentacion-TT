% !TeX root = ../ejemplo.tex

\section{Patrones de arquitectura de software}
Al momento de desarrollar software, es común toparnos con problemáticas que requieren de la toma de decisiones especialmente cuando hablamos sobre cuestiones relacionadas con el diseño de un sistema de software.
Un patrón de arquitectura de software es un conjunto de decisiones tomadas para atacar problemáticas relacionadas con el diseño de un software. Estos incluyen reglas y principios para organizar las interacciones entre subsistemas predefinidos y los roles que estos desempeñan \cite{L07}.

A menudo pueden ser descritos como los \textit{"Planos"} de un sistema, sin embargo, esto no quiere decir que sea la arquitectura final, sino que funcionan como una guía que describe los elementos necesarios para diseñar la arquitectura de la solución a desarrollar, la selección de una arquitectura sobre otra dependerá completamente de los objetivos a alcanzar, los recursos disponibles y la experiencia del equipo de desarrollo \cite{L08}.

Es necesario aclarar que, los patrones de arquitectura no deben confundirse con los patrones de diseño, ya que ambos responden a problemas diferentes durante el desarrollo de un sistema de software, de forma muy breve un patrón de arquitectura describe como crear la lógica de negocio, acceso a los datos, etc. Mientras que los patrones de diseño se usan al implementar estos elementos \cite{L07}.

A continuación, se describen algunos patrones de arquitectura que pueden ser de utilidad al implementar la propuesta de solución para este trabajo terminal.	

\subsection{Arquitectura de capas}

También conocida como \textit{arquitectura de N-capas}, estructura una aplicación en múltiples capas distintas, donde cada una esta encargada de ciertas tareas en especifico, lo que permite dividir un sistema en componentes aislados, lo que facilita el desarrollo rápido de aplicaciones ya que los cambios realizados en una capa no deberían afectar la lógica de las demás \cite{L09}.

Las arquitecturas basadas en este patrón suelen implementar cuatro capas distintas, la capa de presentación, la capa de negocio, de persistencia y de base de datos, sin embargo, y como es de esperarse, la arquitectura de un sistema basado en este patrón puede tener algunas diferencias en el número y tipo de capas que se implementan, por ejemplo, algunas pueden implementar capas de aplicación, servicio o acceso a datos \cite{L07}.

\subsection{Arquitectura orientada a servicios}

Las aplicaciones diseñadas siguiendo este patrón de arquitectura implementan una colección de servicios poco acoplados que se comunican entre sí a través de una red.
Cada uno de los servicios que conforman al sistema se encarga de llevar acabo una función del negocio en específico que después pueden ser requeridos por otro servicio o cliente \cite{L09}.


\subsection{Arquitectura de micro servicios}

Es una arquitectura que combina patrones de diseño para crear múltiples servicios que trabajan de forma independiente y que en conjunto forman la lógica de una aplicación. Es una alternativa a las aplicaciones monolíticas y de la arquitectura basada en servicios, en donde cada servicio esta orientado a implementar partes muy puntuales de la lógica de negocio \cite{L08}.

Su principal ventaja radica en que debido a que sus componentes se encuentran poco acoplados pueden ser desarrollados, desplegados y probados de forma independiente \cite{L09}.

La principal desventaja de este tipo de arquitectura es que puede llegar a ser compleja de implementar ya que requiere de definir la granularidad correcta de los servicios y establecer una comunicación efectiva entre estos \cite{L07}.

\subsection{El patrón modelo vista controlador}

Este patrón divide divide una aplicación en tres componentes interconectados. El modelo, la vista y el controlador. Esta separación permite organizar el código al desacoplar la lógica del negocio, interfaz de usuario y el manejo de las entradas de los usuarios, lo que a su vez promueve la modularidad, mantenibilidad y escalabilidad \cite{L09}. 

El modelo contiene los datos y lógica de negocio de la aplicación. Se encarga de regresar, almacenar y procesar la información.

La vista, también conocida como interfaz de usuario (UI) despliega la información al usuario y responde a las interacciones del usuario.

El controlador funciona como un intermediario entre el modelo y la vista. Se encarga de gestionar las entradas del usuario, actualizar el modelo y la vista para reflejar los cambios realizados en el modelo \cite{L08}.
