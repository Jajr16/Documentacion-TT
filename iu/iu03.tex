% !TeX root = ../ejemplo.tex

%--------------------------------------
\section{IU03 Pantalla Consultar notificaciones}

\subsection{Objetivo}
Permitir que los usuarios puedan gestionar sus notificaciones y marcarlas como leidas.
\subsection{Diseño}
    Esta pantalla \IUref{IU03}{Pantalla Consultar notificaciones } (ver figura~\ref{IU03}) puede ser accedida desde cualquier otra pantalla que no sea el inicio de sesión mediante el botón con forma de campana.
\IUfig[.35]{UI-CU03}{IU03}{Pantalla Consultar notificaciones.}

\IUfig[.8]{UI-CU03_2}{IU03-2}{Pantalla Consultar notificaciones.}
\subsection{Salidas}
Menciona que la notificación seleccionada ha sido establecida como leída.
\subsection{Entradas}
   Ninguna.

\subsection{Comandos}
\begin{itemize}
    \item \IUbutton{Botón con palomita} toma la notificación seleccionada y la marca como leída.
    \item \IUbutton{Buscador} En este buscador se puede buscar las notificaciones por fecha.
    \item \IUbutton{Calendario} Redirige a la pantalla \IUref{UI02}{Consultar calendario escolar}.
    \item \IUbutton{Home} Redirige a la pantalla de bienvenida correspondiente al tipo de usuario.
\end{itemize}

\subsection{Mensajes}
     
\begin{Citemize}
    \item {\bf MSG-8}{``Actualmente no hay notificaciones.}
\end{Citemize}


