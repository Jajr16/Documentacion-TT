% !TeX root = ../ejemplo.tex

%--------------------------------------
\section{IU40 Iniciar Solicitar desbloqueo de cuenta}

\subsection{Objetivo}
    Mandar una petición de desbloqueo de cuenta mediante correo electrónico.

\subsection{Diseño}
	Esta pantalla \IUref{IU40}{Pantalla Solicitar desbloqueo de cuenta} (ver figura~\ref{IU40}) puede ser accedida desde cualquier desde cualquier inicio de sesion.

\IUfig[.35]{UI-CU40}{IU40}{Pantalla Solicitar desbloqueo de cuenta.}

\IUfig[.5]{UI-CU40_2}{IU40-2}{Pantalla Solicitar desbloqueo de cuenta.}

\subsection{Salidas}

    Manda un correo electrónico con una cuenta default con los datos requeridos para solicitar la reactivación de su cuenta.

\subsection{Entradas}
    Una justificación de la causa del bloqueo, para el alumno \hyperlink{Alumno.Boleta}{Boleta} y para el resto de usuarios \hyperlink{Empleado.RFC}{RFC}.


\subsection{Comandos}
\begin{itemize}

    \item \IUbutton{Enviar} Envía un correo electrónico con una cuenta default con los datos requeridos para solicitar la reactivación de su cuenta
    \item \IUbutton{Calendario} Redirige a la pantalla \IUref{UI02}{Consultar calendario escolar}.
    \item \IUbutton{Campana} Redirige a la pantalla \IUref{UI03}{Consultar notificaciones }.
    \item \IUbutton{Home} Redirige a la pantalla de menú correspondiente al tipo de usuario.
	
\end{itemize}

\subsection{Mensajes}

\begin{Citemize}
	\item {\bf MSG1-}{``Los campos no están correctamente llenados.''}
	\item {\bf MSG4-}  ``El proceso no se pudo realizar por un fallo de red.''
	\item {\bf MSG10-}  ``Los datos no coinciden con ningún usuario''.
\end{Citemize}
