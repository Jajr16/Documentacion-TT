% !TeX root = ../ejemplo.tex

\section{Planteamiento del problema}

Actualmente, la inseguridad es una de las principales problemáticas de nuestro país, lo que obliga a las instituciones educativas a destinar recursos significativos para reforzar los protocolos de seguridad, tanto dentro como fuera de sus instalaciones. Estos esfuerzos buscan garantizar la integridad física y emocional de la comunidad educativa y propiciar un entorno seguro donde los estudiantes puedan desarrollar plenamente sus capacidades creativas e intelectuales \cite{L03}.\\

Una de las principales preocupaciones de las instituciones educativas es el acceso no autorizado a sus instalaciones, ya que las brechas en los protocolos de control de acceso pueden derivar en situaciones que comprometen la seguridad de la comunidad escolar \cite{L04}. Según un estudio realizado en los planteles de Iztapalapa y Xochimilco de la Universidad Autónoma Metropolitana, los alumnos han sido víctimas de robos, tanto con violencia como sin ella, frecuentemente atribuidos al ingreso de personas ajenas a la institución. Por ejemplo, en el plantel de Iztapalapa, el 79.3 por ciento del alumnado considera que la inseguridad está relacionada con el acceso de personas externas \cite{L05}.\\

En el caso de la Escuela Superior de Cómputo (ESCOM), la creciente matrícula estudiantil ha presentado desafíos adicionales en materia de seguridad. Entre las problemáticas más frecuentes se encuentra la suplantación de identidad durante los Exámenes a Título de Suficiencia (ETS) y los Exámenes a Título de Suficiencia Especiales. Este problema ocurre cuando personas externas o incluso estudiantes de la institución presentan exámenes en lugar de los alumnos registrados, ya sea mediante credenciales falsificadas o acuerdos entre estudiantes. Estas prácticas no solo afectan la transparencia del proceso educativo, sino que también ponen en riesgo la seguridad de la comunidad escolar al permitir el ingreso de personas cuya intención puede ser desconocida. \\

La falta de un control de acceso eficiente contribuye a este problema, ya que protocolos débiles permiten que las personas ingresen sin una verificación adecuada de su identidad. Actualmente, el ingreso a las instalaciones y a los exámenes en la ESCOM depende de métodos tradicionales que pueden ser fácilmente manipulados. \\