% !TeX root = ../ejemplo.tex

\section{Antecedentes}

En la actualidad, la Escuela Superior de Cómputo (ESCOM) cuenta con diversos mecanismos que permiten evaluar los conocimientos del alumnado, siguiendo un esquema de educación tradicional en el que los alumnos tienen que obtener un mínimo de calificación para acreditar una materia y, donde además, es necesario acreditar todas las materias que componen el plan de estudios de la carrera para poder egresar de esta \cite{L01}.\\

Estos mecanismos de evaluación difieren unos de otros dependiendo del momento en el que se aplican, del valor sobre la calificación final del alumno y de la situación académica de este. Algunos ejemplos incluyen las evaluaciones ordinarias parciales, la entrega de proyectos, evaluaciones extraordinarias, etc. \\

Sin embargo, la \textit{evaluación a título de suficiencia} también conocida como ETS, es un tipo de evaluación especial que se aplica en los planteles del Instituto Politécnico Nacional (IPN) y que permite a los alumnos acreditar materias que no hayan podido acreditar durante los periodos ordinarios y extraordinarios \cite{L02}. \\

A pesar de ser una evaluación que se realiza en todos los planteles del IPN, no existe medio alguno que permita obtener información clara y precisa sobre esta, ya que la poca información que se puede encontrar en la red se limita a instructivos de su inscripción de distintos planteles. \\

Es debido a esto, y a su importancia en el proceso académico de cada una de las instituciones que conforman al IPN que se propuso la realización de este proyecto. \\

 