% !TeX root = ../ejemplo.tex

\section{Propuesta de solución}

Tomando como punto de partida las tecnologías que se describen en capítulo 3 la propuesta de solución a la problemática de la suplantación de la identidad durante las  evaluaciones ETS en la ESCOM consta de dos partes. \\

En primer lugar se desarrollara una aplicación móvil con soporte para dispositivos Android que servirá como una herramienta tanto para el personal de seguridad como para los docentes para reforzar los protocolos de control de acceso a las instalaciones y verificación de la identidad de los alumnos respectivamente. \\

Al funcionar como una herramienta se pretende brindar la funcionalidad necesaria para identificar situaciones en las que los alumnos quieran acceder a las instalaciones aún cuando estos no tienen inscrito algún ETS, en las fechas de aplicación de este.
Por otro lado, también se busca que los docentes puedan tener los elementos necesarios para verificar la identidad de los estudiantes.
Estos dos enfoques de la aplicación móvil serán abordados mediante la implementación de un sistema de consulta de alumnos que se conectara a la base de datos escolar de la institución para verificar los accesos y asistencia a los ETS según sea el caso, en este se mostrara la información necesaria para que tanto el personal de seguridad como los docentes puedan tomar la decisión de permitir el acceso o presentar el examen a los alumnos, esto último también permitirá registrar la asistencia de los alumnos a las evaluaciones para su posterior consulta.\\

Además de esto, el módulo de reconocimiento facial se encuentra pensando para servir como una capa de seguridad adicional que permita a los docentes verificar la identidad de los estudiantes cuando la información que estos obtienen de las consultas al sistema no son suficientes para tomar una decisión.\\

De igual forma la propuesta de solución contará con un módulo de notificaciones e informativa relacionada a los ETS inscritos, asignados y horarios de aplicación.\\

Finalmente, para que la aplicación pueda simular un escenario real se pretende realizar un sistema web para la gestión de los usuarios que componen al sistema es decir, alumnos, docentes, personal de seguridad y de exámenes ETS.






