% !TeX root = ../ejemplo.tex
%--------------------------------------
\section{IUE03 Menú de alumno}

\subsection{Objetivo}
Mostrar una pantalla de home después de iniciar sesión y marcar las acciones que el alumno puede hacer.

\subsection{Diseño}
Esta pantalla \IUref{IUE03}{Pantalla Menú de alumno} (ver figura~\ref{IUE03}) aparece al iniciar sesión exitosamente y muestra las acciones que el alumno puede hacer,ademas de las opciones generales de usuario (Consultar calendario escolar y consultar notificaciones). 

\IUfig[.35]{IUE03}{IUE03}{Pantalla Menú de alumno.}

\subsection{Salidas}

Ninguna.

\subsection{Entradas}
Ninguna.

\subsection{Comandos}
\begin{itemize}
	\item \IUbutton{Consultar periodos de ETS asignados}: Redirige a los alumnos a la pantalla \IUref{IU17}{Consultar periodo de ETS de alumnos}
	\item \IUbutton{Notificaciones}: Redirige a los alumnos a la pantalla \IUref{IU03}{Consultar notificaciones}
	\item \IUbutton{Calendario}: Redirige a los alumnos a la pantalla \IUref{IU02}{Consultar calendario escolar}
	
\end{itemize}

\subsection{Mensajes}

\begin{Citemize}
	\item Ninguno.
\end{Citemize}

