% !TeX root = ../ejemplo.tex

%--------------------------------------
\section{IU01 Pantalla Iniciar sesión de personal escolar móvil}

\subsection{Objetivo}
	Controlar el acceso al sistema mediante una contraseña a fin de que cada usuario acceda solo a las operaciones permitidas para su perfil.

\subsection{Diseño}
	Esta pantalla \IUref{IU01}{Pantalla Iniciar sesión de personal escolar móvil} (ver figura~\ref{IU01}) aparece al iniciar el sistema para los empleados. Para ingresar al mismo se debe escribir el RFC del empleado y la contraseña. 

\IUfig[.35]{UI-CU01}{IU01}{Pantalla Iniciar sesión de personal escolar móvil.}

\subsection{Salidas}

Saludo del sistema y mención de su nombre.

\subsection{Entradas}
RFC y contraseña del empleado.

\subsection{Comandos}
\begin{itemize}
	\item \IUbutton{Entrar}: Verifica que el empleado se encuentre registrado y la contraseña sea la correcta. Si la verificación es correcta, se verifica que tipo de empleado y se muestra la pantalla \IUref{IUE01}{Pantalla de Menús de docente} si es docente o \IUref{IUE02}{Pantalla de Menús de personal de seguridad} si es personal de seguridad.
	
	\item \IUbutton{Presiona aquí para pedir su activación}: Redirige a la pantalla \IUref{UI40}{Pantalla de Solicitar desbloqueo de cuenta}
	
\end{itemize}

\subsection{Mensajes}

\begin{Citemize}
	\item MSG1 Los campos no están correctamente llenados. 
	\item MSG2 Su cuenta esta bloqueada. 
	\item MSG3 El RFC o la contraseña no corresponden con ningún empleado. 
	\item MSG4 El proceso no se pudo realizar por un fallo de red. 
	\item MSG5 Su cuenta ha sido bloqueada por la gran cantidad de intentos de inicio sesión fallidos.
\end{Citemize}

