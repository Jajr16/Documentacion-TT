% !TeX root = ../ejemplo.tex

%--------------------------------------
\begin{UseCase}{CU-41}{Iniciar sesión de personal escolar web}{
		Permitir que solo el personal escolar pueda acceder al sistema, además de separar completamente las funciones del alumno y el personal escolar.
	}
	\UCitem{Versión}{\color{Gray}1}
	\UCitem{Autor}{\color{Gray}Huertas Ramírez Daniel Martín}
	\UCitem{Supervisa}{\color{Gray}De la Cruz de la Cruz Alejandra.}
	\UCitem{Actor}{\hyperlink{Empleado}{Empleado} (\hyperlink{Personal de la DAE}{Personal de la DAE} y \hyperlink{Personal de gestión escolar}{Personal de gestión escolar})}
	\UCitem{Propósito}{Que el empelado pueda acceder al sistema web y sus funciones específicas. }
	\UCitem{Entradas}{\hyperlink{Empleado.RFC}{RFC}, \hyperlink{Empleado.Contraseña}{Contraseña}}
	\UCitem{Origen}{Teclado}
	\UCitem{Salidas}{Saludo del sistema, mención de su nombre.}
	\UCitem{Destino}{Pantalla \IUref{IUE04}{Pantalla de saludo de personal de la DAE} si es un docente o a la pantalla \IUref{IUE05}{saludo Personal de gestión escolar} si es un Personal de gestión escolar.}
	\UCitem{Precondiciones}{El empleado debe estar registrado en el sistema.}
	\UCitem{Postcondiciones}{El empleado accede al sistema y podrá realizar las acciones pertinentes a su cargo.}
	\UCitem{Errores}{
		E1: Cuando falta algún dato requerido entonces el sistema muestra el mensaje {\bf MSG-1}{``Los campos no están correctamente llenados.''}
		
		E2: Cuando la cuenta esta bloqueada el sistema no deja entrar al empleado y muestra el mensaje {\bf MSG-2}``Su cuenta esta bloqueada.''
		
		E3: Cuando la contraseña no corresponde al RFC ingresado el sistema no permite el acceso al empleado y se muestra el mensaje {\bf MSG-3} ``El RFC o la contraseña no corresponden con ningún empleado.''
		
		E4: Cuando se pierde la conexión durante el proceso, los procesos se cancelan y se muestra el mensaje {\bf MSG-4}  ``El proceso no se pudo realizar por un fallo de red.''
		
		E5: Cuando se intenta iniciar varias veces sesión sin éxito la cuenta es bloqueada por seguridad y se muestra el mensaje {\bf MSG-5}  ``Su cuenta ha sido bloqueada por la gran cantidad de intentos de inicio sesión fallidos''.
	}
	\UCitem{Tipo}{Caso de uso primario}
	\UCitem{Observaciones}{}
\end{UseCase}
%--------------------------------------

	\begin{UCtrayectoria}
	\UCpaso[\UCactor] El usuario introduce su RFC y contraseña en el sistema vía la  \IUref{IU33}{Pantalla de Iniciar sesión de personal escolar web}\label{CU41.introduceDatos}.
	\UCpaso[\UCactor] El usuario confirma la operación presionando el botón \IUbutton{Entrar}.
	\UCpaso El sistema verifica que todos los datos requeridos hayan sido capturados.
	\UCpaso El sistema verifica que el empleado este registrado en el sistema.
	\UCpaso El sistema verifica que la cuenta del empleado no este bloqueada.
	\UCpaso El sistema verifica que la contraseña corresponda al RFC.
	\UCpaso El sistema verifica que tipo acceso tiene este empleado.
	\UCpaso La sesión es iniciada con éxito.
	\UCpaso El Empleado es redirigido a la pantalla \IUref{IUE04}{Pantalla de saludo de personal de la DAE} si es personal de la DAE Y \IUref{IUE05}{Pantalla de saludo de personal de gestión escolar} si es personal de gestión escolar.
	
\end{UCtrayectoria}






