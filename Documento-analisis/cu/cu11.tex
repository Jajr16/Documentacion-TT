% !TeX root = ../ejemplo.tex

%--------------------------------------
\begin{UseCase}{CU11}{Mostrar la foto e información del alumno}{

    Permitir que los docentes puedan revisar la información de un alumno especifico que ellos hayan escogido.
}
    \UCitem{Versión}{\color{Gray}1}
    \UCitem{Autor}{\color{Gray}Huertas Ramírez Daniel Martín}
    \UCitem{Supervisa}{\color{Gray}Ulises Vélez Saldaña.}
    \UCitem{Actor}{\hyperlink{Docente }{Docente }}
    \UCitem{Propósito}{Permitir a los docentes revisar que alumnos se presentaran al ETS especifico.}
    \UCitem{Entradas}{Ninguna}
    \UCitem{Origen}{Pantalla}
    \UCitem{Salidas}{Muestra la información del alumno y su foto.}
    \UCitem{Destino}{Ninguno}
    \UCitem{Precondiciones}{El docente debe de haber iniciado sesión.}
    \UCitem{Postcondiciones}{El docente revisa la información del alumno que seleccionó.}
    \UCitem{Errores}{
        E1: Cuando se pierde la conexión durante el proceso, los procesos se cancelan y se muestra el mensaje {\bf MSG4-}  ``El proceso no se pudo realizar por un falló de red.''
    }
    \UCitem{Tipo}{ Extiende de CU01 Iniciar sesión }
    \UCitem{Observaciones}{Ninguna}


\end{UseCase}
%-------------------------------------- 

\begin{UCtrayectoria}

    \UCpaso[\UCactor] El docente accede a la pantalla \IUref{IU09}{Pantalla Foto e información del alumno}\label{CU11.introduceDatos} .
    \UCpaso[\UCactor] El docente revisa los datos del alumno.
    \UCpaso[\UCactor] El docente decide que quiere expandir la foto del alumno para verla mejor presionando el \IUbutton{Ampliar fotografía }.
    \UCpaso El sistema muestra la foto ampliada.
\end{UCtrayectoria}


