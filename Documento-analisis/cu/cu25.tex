% !TeX root = ../ejemplo.tex

%--------------------------------------
\begin{UseCase}{CU26}{Dar de alta de periodo de ETS}{

    Permitir al personal de gestión escolar dar de alta un periodo de ETS.
 }
     \UCitem{Versión}{\color{Gray}1}
     \UCitem{Autor}{\color{Gray}Huertas Ramírez Daniel Martín}
     \UCitem{Supervisa}{\color{Gray}Ulises Vélez Saldaña.}
     \UCitem{Actor}{\hyperlink{Personal de gestión escolar}{Personal de gestión escolar}}
     \UCitem{Propósito}{ Permitir al personal de gestión escolar dar de alta un nuevo periodo de ETS.}
     \UCitem{Entradas}{ \hyperlink{Periodo de ETS.Periodo}{Periodo}, \hyperlink{Periodo-de-ETS.Tipo}{Tipo}, \hyperlink{Periodo-de-ETS.Fecha-de-inicio}{Fecha-de-inicio} y \hyperlink{Periodo-de-ETS.Fecha-de-fin}{Fecha-de-fin}}
     \UCitem{Origen}{Teclado}
     \UCitem{Salidas}{Muestra mensaje {\bf MSG15-} ``Periodo de ETS  dado de alta con éxito''.}
     \UCitem{Destino}{Pantalla \IUref{IU24}{Consultar lista de periodo de ETS } }
     \UCitem{Precondiciones}{El Personal de gestión escolar debe de haber iniciado sesión.}
     \UCitem{Postcondiciones}{El periodo de ETS es dado de alta y guardado en la base de datos}
     \UCitem{Errores}{
 
        E1: Cuando se pierde la conexión durante el proceso, los procesos se cancelan y se muestra el mensaje {\bf MSG4-}  ``El proceso no se pudo realizar por un fallo de red.''
 
        E2: Cuando falta algún dato requerido entonces el sistema muestra el mensaje {\bf MSG1-}{``Los campos no están correctamente llenados.''}

        E3: Cuando el Periodo, Fecha-de-inicio o Fecha-de-fin ya están registradas en el sistema, el proceso no se realiza y se muestra el mensaje {\bf MSG16-}{`` Periodo, Fecha-de-inicio o Fecha-de-fin ya han sido asociadas a un periodo de ETS.''}
     }
     \UCitem{Tipo}{ Extiende de CU41 Iniciar sesión de personal escolar web}
     \UCitem{Observaciones}{Ninguna}
 
 \end{UseCase}
 %-------------------------------------- 
 
 \begin{UCtrayectoria}
 
     \UCpaso[\UCactor] El Personal de gestión escolar accede a la pantalla \IUref{IU25}{ Dar de alta de periodo de ETS }\label{CU25.introduceDatos} e introduce los datos del periodo \hyperlink{Periodo de ETS.Periodo }{Periodo}, \hyperlink{Periodo de ETS.Tipo }{Tipo}, \hyperlink{Periodo de ETS.Fecha-de-inicio }{Fecha-de-inicio} y \hyperlink{Periodo de ETS.Fecha-de-fin }{Fecha-de-fin}.
     \UCpaso[\UCactor] El Personal de gestión escolar oprime el botón \IUbutton{Dar de alta periodo de ETS }.
     \UCpaso El sistema revisa que los datos del periodo sean válidos.
     \UCpaso El sistema verifica que el Periodo, Fecha-de-inicio o Fecha-de-fin no hayan sido registrados con anterioridad.
     \UCpaso El periodo de ETS es dado de alta y guardado en la base de datos.
     \UCpaso[\UCactor] El Personal de gestión escolar es redirigido a la pantalla \IUref{IU24}{Consultar lista de periodo de ETS }.
 
 \end{UCtrayectoria}
 