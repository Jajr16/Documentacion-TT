% \IUref{IUAdmPS}{Administrar Planta de Selección}
% \IUref{IUModPS}{Modificar Planta de Selección}
% \IUref{IUEliPS}{Eliminar Planta de Selección}

%-------------------------------------- COMIENZA descripción del caso de uso.

%\begin{UseCase}[archivo de imágen]{UCX}{Nombre del Caso de uso}{
%--------------------------------------
\begin{UseCase}{CU-05}{Consultar ETS asignados}{
		Este caso de uso permite al docente consultar los ETS que tiene asignados.
	}
	\UCitem{Versión}{\color{Gray}1.0}
	\UCitem{Autor}{\color{Gray}De la cruz De la cruz Alejandra}
	\UCitem{Supervisa}{\color{Gray}Ulises Velez Saldaña}
	\UCitem{Actor}{\hyperlink{Docente}{Docente}}
	\UCitem{Propósito}{Permitir al docente consultar los ETS que le han sido asignados.}
	\UCitem{Entradas}{Selecciona un periodo}
	\UCitem{Origen}{Pantalla táctil}
	\UCitem{Salidas}{Lista de ETS asignados.}
	\UCitem{Destino}{\IUref{IU05}{Pantalla de Consultar ETS}}
	\UCitem{Precondiciones}{El docente debe estar autenticado en el sistema.}
	\UCitem{Postcondiciones}{El docente ha consultado los ETS asignados.}
	\UCitem{Errores}{
			E1: El sistema no puede recuperar la información de los ETS asignados.
			
			E2: No hay ETS asignados al docente.}
	\UCitem{Tipo}{Se extiende del CU01 Iniciar Sesión del docente}
	\UCitem{Observaciones}{Ninguna}
\end{UseCase}
%--------------------------------------
\begin{UCtrayectoria}
	\UCpaso[\UCactor] Selecciona el período académico que desea consultar desde la \IUref{IU04}{Pantalla Periodo de ETS}.
	\UCpaso Verifica que el docente tenga ETS asignados en el periodo seleccionado \Trayref{A}.
	\UCpaso Despliega la lista de ETS asignados al docente en la \IUref{IU05}{Pantalla de Consultar ETS}.
\end{UCtrayectoria}

%--------------------------------------        
\begin{UCtrayectoriaA}{A}{No hay ETS asignados en el periodo seleccionado}
	\UCpaso Muestra un mensaje: {\bf MSG-12-}{``No hay ETS asignados actualmente.''}
	\UCpaso[\UCactor] Presiona el botón \IUbutton{Regresar} para volver a la pantalla anterior.
	\UCpaso Fin de la trayectoria alternativa.
\end{UCtrayectoriaA}
%--------------------------------------        
\begin{UCtrayectoriaA}{B}{Error en la conexión con la base de datos}
	\UCpaso Muestra un mensaje de error: {\bf MSG11-}{``Error al consultar la base de datos. Intente nuevamente más tarde.''}
	\UCpaso[\UCactor] Presiona el botón \IUbutton{Aceptar} para cerrar el mensaje.
	\UCpaso[\UCactor] Puede intentar la consulta nuevamente o presionar el botón \IUbutton{Regresar} para volver a la pantalla anterior.
	\UCpaso Fin de la trayectoria alternativa.
\end{UCtrayectoriaA}

%-------------------------------------- TERMINA descripción del caso de uso.