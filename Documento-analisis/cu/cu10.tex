% \IUref{IUAdmPS}{Administrar Planta de Selección}
% \IUref{IUModPS}{Modificar Planta de Selección}
% \IUref{IUEliPS}{Eliminar Planta de Selección}

%-------------------------------------- COMIENZA descripción del caso de uso.

%\begin{UseCase}[archivo de imágen]{UCX}{Nombre del Caso de uso}{
%--------------------------------------
\begin{UseCase}{CU-10}{Consultar Lista de asistencia a los ETS}{
		Este caso de uso permite al docente visualizar la asistencia de los alumnos inscritos a un ETS asignado.
	}
	\UCitem{Versión}{\color{Gray}1.0}
	\UCitem{Autor}{\color{Gray}De la cruz De la cruz Alejandra}
	\UCitem{Supervisa}{\color{Gray}De la cruz De la cruz Alejandra}
	\UCitem{Actor}{\hyperlink{Docente}{Docente}}
	\UCitem{Propósito}{Permitir al docente registrar la asistencia de los alumnos que estén inscritos a un ETS.}
	\UCitem{Entradas}{-}
	\UCitem{Origen}{Pantalla táctil}
	\UCitem{Salidas}{Confirmación del registro de asistencia de los alumnos.}
	\UCitem{Destino}{\IUref{IU08}{Lista de asistencia de ETS.}}
	\UCitem{Precondiciones}{El docente debe estar autenticado y asignado al ETS correspondiente.}
	\UCitem{Postcondiciones}{La asistencia de los alumnos ha sido registrada en el sistema para el ETS.}
	\UCitem{Errores}{
		\begin{itemize}
			\item El ETS seleccionado no tiene alumnos inscritos.
			\item El sistema pierde la conexión al intentar registrar la asistencia.
		\end{itemize}
	}
	\UCitem{Tipo}{Caso de uso primario}
	\UCitem{Observaciones}{}
\end{UseCase}
%--------------------------------------
\begin{UCtrayectoria}
	\UCpaso[\UCactor] Presiona el botón \IUbutton{Tomar asistencia} desde la pantalla \IUref{IU07}{Pantalla Lista de alumno}.
	\UCpaso Activa la cámara y comienza el proceso de reconocimiento facial de los alumnos presentes. \Trayref{A} \Trayref{B}
	\UCpaso Analiza la imagen de cada alumno y muestra un indicador:
	\begin{itemize}
		\item Verde: El sistema está casi seguro de que la persona es quien dice ser y muestra las características coincidentes.
		\item Rojo: El sistema está casi seguro de que la persona no es quien dice ser y muestra las características que no coinciden. \Trayref{D}
		\item Amarillo: El sistema no está seguro y necesita la ayuda del docente para confirmar la identidad, mostrando tanto las características coincidentes como las que no coinciden. \Trayref{C}
	\end{itemize}
	\UCpaso[\UCactor] Revisa las características del alumno y decide si confirma la identidad del alumno cuando el indicador marque el color Amarillo. 
	\UCpaso Marca la asistencia de cada alumno.
	\UCpaso[\UCactor] Confirma el registro de asistencia.
	\UCpaso Guarda la asistencia y muestra un mensaje de confirmación: {\bf MSG-19-}{``Asistencia registrada exitosamente.''}
	\UCpaso Muestra la lista de asistencia de los alumnos en la \IUref{IU08}{Pantalla Lista de asistencia de ETS.}
\end{UCtrayectoria}
%--------------------------------------        
\begin{UCtrayectoriaA}{A}{El ETS no tiene alumnos inscritos}
	\UCpaso Muestra un mensaje: {\bf MSG-18-}{``No hay alumnos inscritos en este ETS.''}
	\UCpaso[\UCactor] Presiona el botón \IUbutton{Regresar} para volver a la pantalla anterior.
	\UCpaso Fin de la trayectoria alternativa.
\end{UCtrayectoriaA}
%--------------------------------------        
\begin{UCtrayectoriaA}{B}{Error en la conexión con la base de datos}
	\UCpaso[\UCactor] Muestra un mensaje de error: {\bf MSG11-}{``Error al consultar la base de datos. Intente nuevamente más tarde.''}
	\UCpaso[\UCactor] Presiona el botón \IUbutton{Aceptar} para cerrar el mensaje.
	\UCpaso[\UCactor] Puede intentar registrar la asistencia.
	\UCpaso Fin de la trayectoria alternativa.
\end{UCtrayectoriaA}

%-------------------------------------- TERMINA descripción del caso de uso.