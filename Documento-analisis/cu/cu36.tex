% !TeX root = ../ejemplo.tex

%--------------------------------------
\begin{UseCase}{CU36}{Consultar lista de docentes }{

    Permitir al Personal de gestión escolar consultar la lista con la información de docentes registrados.
}
    \UCitem{Versión}{\color{Gray}1}
    \UCitem{Autor}{\color{Gray}Huertas Ramírez Daniel Martín}
    \UCitem{Supervisa}{\color{Gray}Ulises Vélez Saldaña.}
    \UCitem{Actor}{\hyperlink{Personal de gestión escolar}{Personal de gestión escolar}}
    \UCitem{Propósito}{Mostrar una lista con todos los docentes registrados en el sistema.}
    \UCitem{Entradas}{Ninguna}
    \UCitem{Origen}{Teclado}
    \UCitem{Salidas}{Ninguna}
    \UCitem{Destino}{Pantalla \IUref{IU37}{Dar de alta docente}}
    \UCitem{Precondiciones}{ El Personal de gestión escolar debe de haber iniciado sesión}
    \UCitem{Postcondiciones}{Ninguna.}
    \UCitem{Errores}{
        E1: Cuando se pierde la conexión durante el proceso, los procesos se cancelan y se muestra el mensaje {\bf MSG4-}  ``El proceso no se pudo realizar por un fallo de red.''

        E2: Cuando no hay ningún docente dado de alta se muestra el mensaje {\bf MSG14-}  ``No hay docentes dados de alta.''
    }
    \UCitem{Tipo}{ Extiende de CU41 Iniciar sesión de personal escolar web}
    \UCitem{Observaciones}{Ninguna}

\end{UseCase}
%-------------------------------------- 

\begin{UCtrayectoria}
\UCpaso[\UCactor] El Personal de gestión escolar accede a la pantalla \IUref{IU36}{ Consultar lista de docentes}\label{CU36.introduceDatos}.
\UCpaso El sistema muestra la información de todos los docentes.
\UCpaso[\UCactor] El Personal de gestión escolar revisa la información de los docentes dados de alta.
\UCpaso[\UCactor] El Personal de gestión escolar decide que quiere dar de alta a un nuevo docente.
\UCpaso[\UCactor] El Personal de gestión escolar selecciona el botón \IUbutton{Dar de alta docente}.
\UCpaso[\UCactor] El Personal de gestión escolar es redirigido a la pantalla \IUref{IU37}{Dar de alta docente }.
\end{UCtrayectoria}


