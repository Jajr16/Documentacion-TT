%-------------------------------------- COMIENZA descripción del caso de uso.

\begin{UseCase}{CU-21}{Información del Proceso para Presentar ETS}{
		Este caso de uso permite al alumno visualizar la información detallada del proceso para la presentación de los ETS.
	}
	\UCitem{Versión}{\color{Gray}1.0}
	\UCitem{Autor}{\color{Gray}De la cruz De la cruz Alejandra}
	\UCitem{Supervisa}{\color{Gray}Ulises Velez Saldaña}
	\UCitem{Actor}{\hyperlink{Alumno}{Alumno}}
	\UCitem{Propósito}{Permitir al alumno consultar los detalles del proceso para presentar su ETS.}
	\UCitem{Entradas}{-}
	\UCitem{Origen}{Teclado}
	\UCitem{Salidas}{Detalles del proceso para la presentación de ETS}
	\UCitem{Destino}{\IUref{IU18}{Pantalla Detalles del Proceso de ETS}.}
	\UCitem{Precondiciones}{El alumno debe estar autenticado en el sistema y tener ETS asignados.}
	\UCitem{Postcondiciones}{El alumno ha visualizado la información detallada del proceso para presentar su ETS.}
	\UCitem{Errores}{

			E1: Error al recuperar información del proceso debido a fallos en la conexión con la base de datos.

	}
	\UCitem{Tipo}{Caso de uso primario}
	\UCitem{Observaciones}{}
\end{UseCase}
%--------------------------------------
\begin{UCtrayectoria}
	\UCpaso[\UCactor] Selecciona la opción "Información del Proceso para Presentar ETS" desde la \IUref{IUE03}{Pantalla Menú del alumno}.
	\UCpaso Verifica que exista información disponible sobre el proceso para presentar ETS. \Trayref{A}
	\UCpaso Muestra la información detallada del proceso para presentar su ETS.
	\UCpaso[\UCsist] Muestra la información en la \IUref{IU18}{Pantalla Detalles del Proceso de ETS}.
\end{UCtrayectoria}
%--------------------------------------        
\begin{UCtrayectoriaA}{B}{Error al querer mostrar la información}
	\UCpaso Muestra un mensaje de error: {\bf MSG-24-}{``Error al querer mostrar la información. Por favor, intente nuevamente.''}
	\UCpaso[\UCactor] Presiona el botón \IUbutton{Aceptar} para cerrar el mensaje.
	\UCpaso[\UCactor] Decide si intenta acceder a la información nuevamente o presiona el botón \IUbutton{Regresar} para volver a la pantalla principal.
	\UCpaso Fin de la trayectoria alternativa.
\end{UCtrayectoriaA}


