% !TeX root = ../ejemplo.tex

%--------------------------------------
\begin{UseCase}{CU-xx}{Iniciar sesión de alumnos}{
		Permitir que alumno pueda acceder al sistema, además de separar completamente las funciones de el alumno y el personal escolar.
	}
	\UCitem{Versión}{\color{Gray}1}
	\UCitem{Autor}{\color{Gray}Huertas Ramírez Daniel Martín}
	\UCitem{Supervisa}{\color{Gray}Ulises Vélez Saldaña.}
	\UCitem{Actor}{\hyperlink{Alumno}{Alumno}}
	\UCitem{Propósito}{Que el alumno pueda acceder al sistema móvil y sus funciones específicas. }
	\UCitem{Entradas}{\hyperlink{Alumno.Boleta}{Boleta}, \hyperlink{Alumno.Contraseña}{Contraseña}}
	\UCitem{Origen}{Teclado}
	\UCitem{Salidas}{Saludo del sistema, mención de su nombre.}
	\UCitem{Destino}{Pantalla \IUref{IUE03}{Pantalla de Menú de alumnos}}
	\UCitem{Precondiciones}{El alumno debe estar registrado en la ESCOM.}
	\UCitem{Postcondiciones}{El alumno accede al sistema y podrá realizar las acciones pertinentes a su cargo.}
	\UCitem{Errores}{
		E1: Cuando falta algún dato requerido entonces el sistema muestra el mensaje {\bf MSG1-}{``Los campos no están correctamente llenados.''}
		
		E2: Cuando la cuenta esta bloqueada el sistema no deja entrar al alumno y muestra el mensaje {\bf MSG2-}``Su cuenta esta bloqueada.''
		
		E3: Cuando la contraseña no corresponde a la boleta ingresado el sistema no permite el acceso al alumno y se muestra el mensaje {\bf MSG6-} ``La boleta o la contraseña no corresponden con ningún alumno.''
		
		E4: Cuando se pierde la conexión durante el proceso, los procesos se cancelan y se muestra el mensaje {\bf MSG4-}  ``El proceso no se pudo realizar por un fallo de red.''
		
		E5: Cuando se intenta iniciar varias veces sesión sin éxito la cuenta es bloqueada por seguridad y se muestra el mensaje {\bf MSG5-}  ``Su cuenta ha sido bloqueada por la gran cantidad de intentos de inicio sesión fallidos''.
	}
	\UCitem{Tipo}{Caso de uso primario}
	\UCitem{Observaciones}{}
\end{UseCase}
%--------------------------------------

\begin{UCtrayectoria}
	\UCpaso[\UCactor] Introduce su boleta y contraseña en el sistema vía la  \IUref{IU13}{Pantalla de Iniciar sesión de alumno escolar móvil}\label{CU16.introduceDatos}.
	\UCpaso[\UCactor] Confirma la operación presionando el botón \IUbutton{Entrar}.
	\UCpaso Verifica que todos los datos requeridos hayan sido capturados.
	\UCpaso Verifica que el alumno este registrado en el sistema.
	\UCpaso Verifica que la cuenta del alumno no este bloqueada.
	\UCpaso Verifica que la contraseña corresponda a la boleta.
	\UCpaso Verifica que tipo acceso tiene este alumno.
	\UCpaso La sesión es iniciada con éxito.
	\UCpaso El alumno es redirigido a la pantalla \IUref{IUE03}{Pantalla de Menú de alumnos}.
	
\end{UCtrayectoria}







