% !TeX root = ../ejemplo.tex

%--------------------------------------
\begin{UseCase}{CU-29}{Dar de alta ETS }{
    Permitir al personal de gestión escolar dar de alta un nuevo ETS.
    
     }
         \UCitem{Versión}{\color{Gray}1}
         \UCitem{Autor}{\color{Gray}Huertas Ramírez Daniel Martín}
         \UCitem{Supervisa}{\color{Gray}De la cruz De la cruz Alejandra.}
         \UCitem{Actor}{\hyperlink{Personal de gestión escolar}{Personal de gestión escolar}}
         \UCitem{Propósito}{Permitir que el personal de gestión escolar dar de alta un nuevo ETS relacionado con el periodo de ETS actual.}
         \UCitem{Entradas}{ \hyperlink{ETS.ETS }{ETS},  \hyperlink{ETS.Periodo -de-ETS }{ Periodo -de-ETS},  \hyperlink{ETS.Fecha}{Fecha},  \hyperlink{ETS.Turno}{Turno},  \hyperlink{ETS.Cupo} {Cupo} ,  \hyperlink{ETS.Unidad-de-aprendizaje }{Unidad-de-aprendizaje},  \hyperlink{ETS.Salon}{Salon} y \hyperlink{ETS.Docente}{Docente} .}
         \UCitem{Origen}{Teclado}
         \UCitem{Salidas}{Muestra mensaje {\bf MSG-35} ``ETS  dado de alta con éxito''.}
         \UCitem{Destino}{Pantalla \IUref{IU26}{Consultar lista de ETS} }
         \UCitem{Precondiciones}{El Personal de gestión escolar debe de haber iniciado sesión.}
         \UCitem{Postcondiciones}{El ETS es dado de alta y guardado en la base de datos}
         \UCitem{Errores}{
     
            E1: Cuando se pierde la conexión durante el proceso, los procesos se cancelan y se muestra el mensaje {\bf MSG-28}  ``El proceso no se pudo realizar por un fallo de red.''
     
            E2: Cuando falta algún dato requerido entonces el sistema muestra el mensaje {\bf MSG-29}{``Los campos no están correctamente llenados.''}
    
            E3: Cuando el dato ETS o salón ya están registradas en el sistema, el proceso no se realiza y se muestra el mensaje {\bf MSG-36}{`` ETS o salón  ya han sido asociadas a un ETS de ETS.''}
         }
         \UCitem{Tipo}{ Extiende de CU28 Consultar lista de ETS}
         \UCitem{Observaciones}{Ninguna}
     
     \end{UseCase}
     %-------------------------------------- 
     
     \begin{UCtrayectoria}
     
         \UCpaso[\UCactor] El Personal de gestión escolar accede a la pantalla \IUref{IU27}{ Dar de alta ETS }\label{CU29.introduceDatos} desde la pantalla \IUref{IU26}{Consultar lista de ETS} apretando el botón \IUbutton{Dar de alta un ETS} e introduce los datos del ETS {\hyperlink{ETS.ETS }{ETS},  \hyperlink{ETS.Periodo -de-ETS }{ Periodo -de-ETS},  \hyperlink{ETS.Fecha}{Fecha},  \hyperlink{ETS.Turno}{Turno},  \hyperlink{ETS.Cupo} {Cupo} ,  \hyperlink{ETS.Unidad-de-aprendizaje }{Unidad-de-aprendizaje},  \hyperlink{ETS.Salon}{Salon} y \hyperlink{ETS.Docente}{Docente} .
         \UCpaso[\UCactor] El Personal de gestión escolar oprime el botón \IUbutton{Dar de alta ETS }.
         \UCpaso El sistema revisa que los datos del ETS sean válidos.
         \UCpaso El sistema verifica que ETS o salón no hayan sido registrados con anterioridad.
         \UCpaso EL ETS es dado de alta y guardado en la base de datos.
         \UCpaso[\UCactor] El Personal de gestión escolar es redirigido a la pantalla \IUref{IU26} Consultar lista de ETS}.
     
     \end{UCtrayectoria}
     
    