\section{Sistema de control de acceso biométrico mediante reconocimiento facial con técnicas de vivacidad}

\section{Descripción}

El proyecto es un sistema de control de acceso biométrico basado en reconocimiento facial, diseñado para autenticar usuarios en tiempo real utilizando técnicas de detección de vivacidad, evitando fraudes como el uso de fotografías o videos. Este sistema, implementado en entornos como instituciones educativas y empresas, se enfoca en garantizar un acceso seguro y eficiente. Sus principales características incluyen una arquitectura modular, integración sencilla y énfasis en la privacidad de los datos, haciéndolo ideal para escenarios que requieren altos estándares de seguridad.

\section{Ventajas}

\begin{itemize}
    \item Ofrece alta seguridad al utilizar reconocimiento facial con técnicas de vivacidad, evitando fraudes mediante fotografías o videos.
    \item Permite autenticación en tiempo real, mejorando la eficiencia de los procesos de acceso.
    \item Su arquitectura modular facilita la implementación en distintos entornos.
    \item Priorización de la privacidad de los datos, cumpliendo con estándares de seguridad.
\end{itemize}

\section{Desventajas}

\begin{itemize}
    \item Puede requerir hardware especializado, como cámaras avanzadas, incrementando costos.
    \item La detección de vivacidad podría fallar en condiciones de iluminación deficientes.
    \item Dependencia de datos biométricos, lo que implica riesgos si no se protege adecuadamente.
\end{itemize}