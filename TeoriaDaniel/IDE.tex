% !TeX root = ../ejemplo.tex

\section{IDE}

Un entorno de desarrollo integrado (IDE) es un sistema de software para el diseño de aplicaciones que combina herramientas del desarrollador comunes en una sola interfaz gráfica de usuario (GUI). Generalmente, un IDE cuenta con las siguientes características [28]:

\begin{list}{}%
    {\setlength{\leftmargin}{1cm}%
     \setlength{\rightmargin}{1cm}%
     \setlength{\itemsep}{0.5\baselineskip}%
     \setlength{\parsep}{0pt}}
     
    \item\relax
    \small
    \textbf{Editor de código fuente:} editor de texto que ayuda a escribir el código de software con funciones como el resaltado de la sintaxis con indicaciones visuales, el relleno automático específico para el lenguaje y la comprobación de errores a medida que se escribe el código [28].
    \textbf{Automatización de las compilaciones locales:} herramientas que automatizan las tareas sencillas y repetitivas como parte de la creación de una compilación local del software para que use el desarrollador, como la compilación del código fuente de la computadora en código binario, el empaquetado de ese código y la ejecución de pruebas automatizadas [28].
    \textbf{Depurador:} programa que sirve para probar otros programas y mostrar la ubicación de un error en el código original de forma gráfica [28].

\end{list}

Para nuestro proyecto ocuparemos el IDE llamado android studio.

\subsection{Android Studio}

\begin{list}{}%
    {\setlength{\leftmargin}{1cm}\setlength{\rightmargin}{1cm}}
    \item\relax
    \small

Android Studio es el entorno de desarrollo integrado (IDE) oficial para el desarrollo de aplicaciones del mismo OS. Se basa en IntelliJ IDEA, un entorno de desarrollo integrado de Java para software, e incorpora sus herramientas de desarrollo y edición de código [29].

Para respaldar el desarrollo de aplicaciones dentro del sistema operativo, Android Studio utiliza un sistema de compilación basado en Gradle, un emulador de Android, plantillas de código e integración de GitHub. Cada proyecto en Android Studio tiene una o más modalidades con código fuente y archivos de recursos [29].

\end{list}


\subsection{Gradle}

\begin{list}{}%
    {\setlength{\leftmargin}{1cm}\setlength{\rightmargin}{1cm}}
    \item\relax
    \small

Gradle, es una herramienta que permite la automatización de compilación de código abierto, la cual se encuentra centrada en la flexibilidad y el rendimiento. Los scripts de compilación de Gradle se escriben utilizando Kotlin DSL (Domain Specific Language) [30].
Gradle tiene una gran flexibilidad y nos deja hacer usos de otros lenguajes y no solo de Java, también cuenta con un sistema de gestión de dependencias muy estable. Gradle es altamente personalizable y rápido ya que completa las tareas de forma rápida y precisa reutilizando las salidas de las ejecuciones anteriores, sólo procesar las entradas que presentan cambios en paralelo [30].
Además es el sistema de compilación oficial para Android y cuenta con soporte para diversas tecnologías y lenguajes [30].

\end{list}


\subsection{GitHub}

\begin{list}{}%
    {\setlength{\leftmargin}{1cm}\setlength{\rightmargin}{1cm}}
    \item\relax
    \small

GitHub es una plataforma de desarrollo colaborativo que aloja proyectos en la nube utilizando el sistema de control de versiones llamado Git. Ayuda a los desarrolladores a almacenar y administrar el código llevando un registro de cambios. Generalmente el código es abierto, lo que permite realizar proyectos compartidos y mantener el seguimiento detallado de su progreso. La plataforma GitHub también funciona como red social conectando a los desarrolladores con los usuarios. Como usuario puedes descargar programas o aplicaciones, y de la misma manera puedes aportar a su desarrollo ofreciendo mejoras y discutir las cuestiones que te interesan en foros temáticos [31].
\end{list}

\subsection{Git}

\begin{list}{}%
    {\setlength{\leftmargin}{1cm}\setlength{\rightmargin}{1cm}}
    \item\relax
    \small

Git es un software de control de versiones diseñado por Linus Torvalds, el creador de Linux. El propósito de Git es llevar un registro de cambios y coordinar el trabajo de varias personas en un repositorio compartido. Desde su creación en 2005, este software llegó a convertirse en uno de los VCS más populares: según la encuesta de Stack Overflow (en inglés), más del noventa porciento de los desarrolladores usan Git en sus proyectos [30].
Git proporciona herramientas para un trabajo rápido y eficiente dentro de un equipo. El control de versiones permite a los desarrolladores descargar una copia del código fuente a sus repositorios locales (PC), realizar cambios y subir una versión nueva al repositorio compartido. Todas las modificaciones se guardan en versiones independientes, sin afectar el archivo original. Se pueden comparar cambios realizados, ver quién modificó el código y determinar en qué momento se introdujo un error para poder revertirlo. De esta forma todos los desarrolladores interesados en el proyecto tienen acceso al historial de modificaciones realizadas y pueden contribuir mejorando el código del software [31].

\end{list}

Esta última es una herramienta que tenemos planeada usar para el control de versiones en nuestro proyecto. 
