%=========================================================
\chapter{Modelo del Negocio}	
\label{cap:reqSist}

	En este capítulo se modela la {\em Arquitectura del negocio} la cual está conformada por la Ontología del negocio ({\em Términos} y {\em Hechos del negocio}), Arquitectura de procesos y las {\em Reglas del negocio}. Primero se especifica brevemente el {\em Contexto} en el que los términos tienen significado.
	
	En las secciones \ref{sec:terminosDeNegocio} y \ref{sec:hechosDeNegocio} se presentan los Términos del negocio a manera de Glosario y por último se presentan los Hechos del negocio a manera de relaciones entre términos del negocio.

%----------------------------------------------------------
\section{Contexto}

	\cdtInstrucciones{El contexto debe explicar bajo que ambiente los términos del negocio son aplicables y proporcionar información general para su comprensión inicial.\\}
	
	
%---------------------------------------------------------
\section{Términos del Negocio}
\label{sec:terminosDeNegocio}

\begin{description}
	% Ejemplo de un término literal.
	 \item[\hypertarget{tUnidadAcademica}{Unidad académica:}] Se refiere a la institución educativa en donde los usuarios se desenvuelven diariamente.
	
	\item[\hypertarget{tUnidadAprendizaje}{Unidad de aprendizaje:}] Son los elementos que componen un plan de estudios de alguna de las carreras ofertadas en la \hyperlink{tUnidadAcademica}{unidad académica}. Es necesario que los alumnos acrediten todas sus materias para continuar con su formación académica.
	
	\item[\hypertarget{tETS}{Exámen a Título de Suficiencia (ETS):}] Prueba final que permite a los alumnos acreditar una materia reprobada, y para la cual se requiere verificación de identidad.
	
	\item[\hypertarget{tAlumno}{Alumno:}] (es un tipo de Usuario) Se refiere a las personas inscritas dentro de algún plan de estudios ofertado en la \hyperlink{tUnidadAcademica}{unidad académica}.
	
	\item[\hypertarget{tDocenteAplicador}{Docente aplicador:}] (es un tipo de Usuario) Se refiere a las personas registradas como trabajadores que dan clases a los alumnos y supervisan los ETS asignados.
	
	\item[\hypertarget{tPersonalSeguridad}{Personal de seguridad:}] (es un tipo de Usuario) Se refiere a las personas registradas como trabajadores y que permiten o no el acceso a la \hyperlink{tUnidadAcademica}{unidad académica}.
	
	\item[\hypertarget{tCodigoQR}{Código QR:}] Código único generado por el sistema que permite resolver tareas de control de acceso a las instalaciones y a servicios de autenticación.
	
	\item[\hypertarget{tSistemaVerificacion}{Sistema de verificación de la identidad:}] Conjunto de procesos que permiten validar la identidad de los alumnos que buscan aplicar un ETS.
	
	\item[\hypertarget{tCredencialEscolar}{Credencial escolar:}] Documento con datos de identificación que pueden usarse junto a los registros de inscripción a ETS para permitir o no el acceso a la \hyperlink{tUnidadAcademica}{unidad académica}.
	
	\item[\hypertarget{tControlAcceso}{Control de acceso:}] Sistema implementado para verificar y autorizar el acceso a la \hyperlink{tUnidadAcademica}{unidad académica}.
	
	\item[\hypertarget{tRegistroAcceso}{Registro de acceso:}] Historial digital que documenta los accesos permitidos y denegados, incluyendo datos de cada intento de entrada para consulta posterior.
	
%	\brTermSensor{tVelocimetro}{Velocímetro:}{Velocidad de un Vehículo.}{Kilometros/hora.}{Constantemente siempre que el \cdtRef{tVehiculo}{vehículo} esté encendido.}
\end{description}
%---------------------------------------------------------
\section{Modelo del dominio del problema}
\label{sec:terminosDeNegocio}
\subsection{Modelo del dominio del problema}

El modelo del dominio del problema se muestra en la figura~\ref{fig:modeloDeDominio}, a continuación se describen cada una de las entidades y sus relaciones.

\begin{figure}[htbp!]
	\begin{center}
		\includegraphics[angle=90,width=.95\textwidth]{images/modeloDelDominioDelProblema}
		\caption{Modelo del dominio del problema}
		\label{fig:modeloDeDominio}
	\end{center}
\end{figure}

\begin{cdtEntidad}{Alumno}{Alumno}
	\brAttr{registro}{Registro}{Id}{Número de registro utilizado para identificar un alumno}{Sí}
	\brAttr{nombre}{Nombre}{Palabra Corta}
	{Nombre o nombres del alumno.}{Sí}
	\brAttr{primerApellido}{Primer apellido}{Palabra Corta}
	{Primer apellido del alumno.}{Sí}
	\brAttr{segundoApellido}{Segundo apellido}{Palabra Corta}
	{Segundo apellido del alumno.}{No}
	\brAttr{CURP}{CURP}{CURP}
	{CURP del alumno.}{Sí}
	\brAttr{nacimiento}{Nacimiento}{Fecha}
	{Fecha de nacimiento del alumno.}{Sí}
	\brAttr{genero}{Género}{Domicilio}
	{Género del alumno.}{No}
	\brAttr{telefono}{Teléfono}{Telefono}
	{Teléfono para contactar al alumno.}{Sí}
	\brAttr{correo}{Correo}{Correo}
	{Correo del alumno para enviar información académica y escolar y para recuperación de clave de acceso.}{Sí}
	\cdtEntityRelSection
	\brRel{\brRelComposition}{Domicilio}{Un \hyperlink{Alumno}{Alumno} reside en un \hyperlink{Domicilio}{Domicilio}}	
	\brRel{\brRelAgregation}{Grupo}{Un \hyperlink{Alumno}{Alumno} toma un \hyperlink{Curso}{Curso}}	
\end{cdtEntidad}

%- - - - - - - - - - - - - - - - - - - - - - - - - - - - - 
\begin{cdtEntidad}{AlumnoExtranjero}{Alumno Extranjero}%{}
	\brAttr{numeroResidente}{Numero de residente}{Id}{Número de registro dado por la Secretaría de Relaciones Exteriores a los extranjeros.}{Si}
	\brAttr{paisOrigen}{Pais origen}{\hyperlink{Pais}{País}}
	{País de origen del alumno extranjero.}{Sí}
	\cdtEntityRelSection
	\brRel{\brRelAgregation}{País}{Un \hyperlink{Alumno}{Alumno} es originario de un \hyperlink{Pais}{Pais}}	
	\brRel{\brRelGeneralization}{Alumno}{Un \hyperlink{AlumnoExtranjero}{Alumno Extranjero} es un  \hyperlink{Alumno}{Alumno}}	
\end{cdtEntidad}

%---------------------------------------------------------
\section{Modelado de Reglas de negocio}

\input{3-1-reglas}

