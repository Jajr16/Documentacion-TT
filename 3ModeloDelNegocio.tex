%=========================================================
\chapter{Modelo del Negocio}	
\label{cap:reqSist}

	En este capítulo se modela la {\em Arquitectura del negocio} la cual está conformada por la Ontología del negocio ({\em Términos} y {\em Hechos del negocio}), Arquitectura de procesos y las {\em Reglas del negocio}. Primero se especifica brevemente el {\em Contexto} en el que los términos tienen significado.
	
	En las secciones \ref{sec:terminosDeNegocio} y \ref{sec:hechosDeNegocio} se presentan los Términos del negocio a manera de Glosario y por último se presentan los Hechos del negocio a manera de relaciones entre términos del negocio.

%----------------------------------------------------------
\section{Contexto}

	\cdtInstrucciones{El contexto debe explicar bajo que ambiente los términos del negocio son aplicables y proporcionar información general para su comprensión inicial.\\}
	
	
%---------------------------------------------------------
\section{Términos del Negocio}
\label{sec:terminosDeNegocio}

\begin{description}
	% Ejemplo de un término literal.
	 \item[\hypertarget{tUnidadAcademica}{Unidad académica:}] Se refiere a la institución educativa en donde los usuarios se desenvuelven diariamente.
	
	\item[\hypertarget{tUnidadAprendizaje}{Unidad de aprendizaje:}] Son los elementos que componen un plan de estudios de alguna de las carreras ofertadas en la \hyperlink{tUnidadAcademica}{unidad académica}. Es necesario que los alumnos acrediten todas sus materias para continuar con su formación académica.
	
	\item[\hypertarget{tETS}{Exámen a Título de Suficiencia (ETS):}] Prueba final que permite a los alumnos acreditar una materia reprobada, y para la cual se requiere verificación de identidad.
	
	\item[\hypertarget{tAlumno}{Alumno:}] (es un tipo de Usuario) Se refiere a las personas inscritas dentro de algún plan de estudios ofertado en la \hyperlink{tUnidadAcademica}{unidad académica}.
	
	\item[\hypertarget{tDocenteAplicador}{Docente aplicador:}] (es un tipo de Usuario) Se refiere a las personas registradas como trabajadores que dan clases a los alumnos y supervisan los ETS asignados.
	
	\item[\hypertarget{tPersonalSeguridad}{Personal de seguridad:}] (es un tipo de Usuario) Se refiere a las personas registradas como trabajadores y que permiten o no el acceso a la \hyperlink{tUnidadAcademica}{unidad académica}.
	
	\item[\hypertarget{tCodigoQR}{Código QR:}] Código único generado por el sistema que permite resolver tareas de control de acceso a las instalaciones y a servicios de autenticación.
	
	\item[\hypertarget{tSistemaVerificacion}{Sistema de verificación de la identidad:}] Conjunto de procesos que permiten validar la identidad de los alumnos que buscan aplicar un ETS.
	
	\item[\hypertarget{tCredencialEscolar}{Credencial escolar:}] Documento con datos de identificación que pueden usarse junto a los registros de inscripción a ETS para permitir o no el acceso a la \hyperlink{tUnidadAcademica}{unidad académica}.
	
	\item[\hypertarget{tControlAcceso}{Control de acceso:}] Sistema implementado para verificar y autorizar el acceso a la \hyperlink{tUnidadAcademica}{unidad académica}.
	
	\item[\hypertarget{tRegistroAcceso}{Registro de acceso:}] Historial digital que documenta los accesos permitidos y denegados, incluyendo datos de cada intento de entrada para consulta posterior.
	
%	\brTermSensor{tVelocimetro}{Velocímetro:}{Velocidad de un Vehículo.}{Kilometros/hora.}{Constantemente siempre que el \cdtRef{tVehiculo}{vehículo} esté encendido.}
\end{description}
%---------------------------------------------------------
\section{Modelo del dominio del problema}
\label{sec:terminosDeNegocio}
\subsection{Modelo del dominio del problema}

El modelo del dominio del problema se muestra en la figura~\ref{fig:modeloDeDominio}, a continuación se describen cada una de las entidades y sus relaciones.

\begin{figure}[htbp!]
	\begin{center}
		\includegraphics[angle=90,width=.95\textwidth]{images/DER}
		\caption{Modelo del dominio del problema}
		\label{fig:modeloDeDominio}
	\end{center}
\end{figure}
%---------------------------------------------------------
\begin{cdtEntidad}{ESCOM}{ESCOM}
	\brAttr{Id\_escuela}{Id\_escuela}{Id}{Número de registro usado para identificar a la escuela.}{Sí}
	\brAttr{nombre}{Nombre}{Palabra}
	{Nombre de la escuela}{Sí}
	\brAttr{ubicacion}{Ubicación}{Frase}
	{Ubicación en la que se encuentra la escuela.}{Sí}
	\brAttr{telefono}{Teléfono}{Teléfono}
	{Teléfono para contactar a la escuela.}{Si}
%- - - - - - - - - - - - - - - - - - - - - - - - - - - - - 
	\cdtEntityRelSection
	\brRel{\brRelComposition}{Programa\_académico}{ \hyperlink{ESCOM}{ESCOM} está compuesto por un \hyperlink{PA}{Programa\_académico}}	
	\brRel{\brRelAgregation}{Persona}{Una \hyperlink{Persona}{Persona} pertenece a \hyperlink{ESCOM}{ESCOM}}	
\end{cdtEntidad}
%---------------------------------------------------------
\begin{cdtEntidad}{PA}{Programa\_académico}
	\brAttr{Id\_PA}{Id\_PA}{Id}{Número de registro usado para identificar al programa académico.}{Sí}
	\brAttr{nombre}{Nombre}{Palabra}
	{Nombre del programa académico}{Sí}
	\brAttr{descripcion}{Descripción}{Frase}
	{Descripción que habla sobre el programa académico.}{Sí}
	\brAttr{Id\_escuela}{Id\_escuela}{Id}
	{Id de la escuela a la que pertenece el programa académico.}{Si}
%- - - - - - - - - - - - - - - - - - - - - - - - - - - - -
	\cdtEntityRelSection
	\brRel{\brRelComposition}{Unidad\_aprendizaje}{Un \hyperlink{PA}{Programa\_academico} está compuesto por una \hyperlink{UA}{Unidad\_aprendizaje}}	
	\brRel{\brRelAgregation}{ESCOM}{\hyperlink{ESCOM}{ESCOM} está compuesto por un \hyperlink{PA}{Programa\_académico}}	
\end{cdtEntidad}
%---------------------------------------------------------
\begin{cdtEntidad}{Persona}{Persona}
	\brAttr{CURP}{CURP}{Id}{Código usado para identificar a las personas.}{Sí}
	\brAttr{nombre}{Nombre}{Frase larga}
	{Nombre de la persona.}{Sí}
	\brAttr{sexo}{Sexo}{Caracter}
	{Letra que servirá para identificar el sexo de un alumno ('M' para masculino, 'F' para femenino).}{Sí}
	\brAttr{Id\_escuela}{Id\_escuela}{Id}
	{Id de la escuela a la que pertenece la persona.}{Si}
	%- - - - - - - - - - - - - - - - - - - - - - - - - - - - -
	\cdtEntityRelSection
	\brRel{\brRelComposition}{ESCOM}{Una \hyperlink{Persona}{Persona} pertenece a \hyperlink{ESCOM}{ESCOM}}	
	\brRel{\brRelAgregation}{Docente}{Una \hyperlink{Persona}{Persona} es un \hyperlink{Docente}{Docente}}	
	\brRel{\brRelAgregation}{Personal\_Seguridad}{Una \hyperlink{Persona}{Persona} es un \hyperlink{PS}{Personal\_Seguridad}}
	\brRel{\brRelAgregation}{Alumno}{Una \hyperlink{Persona}{Persona} es un  \hyperlink{Alumno}{Alumno}}
	\brRel{\brRelAgregation}{Usuario}{Una \hyperlink{Persona}{Persona} cuenta con un \hyperlink{Usuario}{Usuario}}
\end{cdtEntidad}
%---------------------------------------------------------
\begin{cdtEntidad}{Docente}{Docente}
	\brAttr{CURP}{CURP}{Id}{Código usado para identificar a las personas.}{Sí}
	\brAttr{RFC}{RFC}{Id}
	{RFC que identifica al Docente.}{Sí}
	\brAttr{CorreoI}{CorreoI}{Correo}
	{Correo institucional del docente.}{Sí}
	%- - - - - - - - - - - - - - - - - - - - - - - - - - - - -
	\cdtEntityRelSection
	\brRel{\brRelComposition}{CargoDocente}{Un \hyperlink{Docente}{Docente} tiene asignado un \hyperlink{Cargo}{Cargo}}
	\brRel{\brRelComposition}{ETS}{Un \hyperlink{Docente}{Docente} \hyperlink{Aplica}{Aplica} un \hyperlink{ETS}{ETS}}
	\brRel{\brRelComposition}{Persona}{Un \hyperlink{Docente}{Docente} es una \hyperlink{Persona}{Persona}}
\end{cdtEntidad}
%---------------------------------------------------------
\begin{cdtEntidad}{CargoDocente}{CargoDocente}
	\brAttr{RFC}{RFC}{Id}
	{RFC que identifica al Docente.}{Sí}
	\brAttr{IdCargo}{IdCargo}{Id}{Código usado para identificar un cargo dentro de la escuela.}{Sí}
\end{cdtEntidad}
%---------------------------------------------------------
\begin{cdtEntidad}{Cargo}{Cargo}
	\brAttr{IdCargo}{IdCargo}{Id}{Código usado para identificar un cargo dentro de la escuela.}{Sí}
	\brAttr{Cargo}{Cargo}{Frase corta}
	{Nombre del cargo existente dentro de la institución escolar.}{Sí}
	%- - - - - - - - - - - - - - - - - - - - - - - - - - - - -
	\cdtEntityRelSection
	\brRel{\brRelComposition}{CargoDocente}{Un \hyperlink{Cargo}{Cargo} es asignado a un \hyperlink{Docente}{Docente}}
\end{cdtEntidad}
%---------------------------------------------------------
\begin{cdtEntidad}{PS}{Personal\_Seguridad}
	\brAttr{CURP}{CURP}{Id}{Código usado para identificar a las personas.}{Sí}
	\brAttr{Turno}{Turno}{Caracter}
	{Letra usada identificar el turno en el que se aplica el ETS ('M' para matutino, 'V' para vespertino).}{Sí}
	\brAttr{Cargo}{Cargo}{Frase corta}
	{Nombre del cargo del personal de seguridad.}{Sí}
	%- - - - - - - - - - - - - - - - - - - - - - - - - - - - -
	\cdtEntityRelSection
	\brRel{\brRelComposition}{Persona}{Un \hyperlink{PS}{Personal\_Seguridad} es una \hyperlink{Persona}{Persona}}
\end{cdtEntidad}
%---------------------------------------------------------
\begin{cdtEntidad}{Alumno}{Alumno}
	\brAttr{Boleta}{Boleta}{Id}
	{Código usado para identificar al alumnado de la institución}{Sí}
	\brAttr{CURP}{CURP}{Id}{Código usado para identificar a las personas.}{Sí}
	\brAttr{CorreoI}{CorreoI}{Correo}
	{Correo institucional del docente.}{Sí}
	%- - - - - - - - - - - - - - - - - - - - - - - - - - - - -
	\cdtEntityRelSection
	\brRel{\brRelComposition}{Persona}{Un \hyperlink{Alumno}{Alumno} es una \hyperlink{Persona}{Persona}}
	\brRel{\brRelComposition}{ETS}{Un \hyperlink{Alumno}{Alumno} tiene una \hyperlink{InscripcionETS}{inscripción} a un \hyperlink{ETS}{ETS}}
\end{cdtEntidad}
%---------------------------------------------------------
\begin{cdtEntidad}{Usuario}{Usuario}
	\brAttr{Usuario}{Usuario}{Usuario}{Nombre de usuario asignado a una persona dentro del sistema.}{Sí}
	\brAttr{Password}{Password}{Contraseña}
	{Contraseña ligada al usuario de una persona registrada dentro del sistema.}{Sí}
	\brAttr{TipoU}{TipoU}{Id}
	{Número que identificará a los tipos de usuario registrados dentro del sistema.}{Sí}
	\brAttr{CURP}{CURP}{Id}
	{Código usado para identificar a las personas.}{Sí}
	%- - - - - - - - - - - - - - - - - - - - - - - - - - - - -
	\cdtEntityRelSection
	\brRel{\brRelComposition}{Persona}{Un \hyperlink{Usuario}{Usuario} es asignado a una \hyperlink{Persona}{Persona}}
	\brRel{\brRelComposition}{Tipo\_Usuario}{Un \hyperlink{Usuario}{Usuario} tiene un \hyperlink{TU}{Tipo\_Usuario}}
\end{cdtEntidad}
%---------------------------------------------------------
\begin{cdtEntidad}{TU}{Tipo\_Usuario}
	\brAttr{Id\_TU}{Id\_TU}{Id}
	{RFC que identifica al Docente.}{Sí}
	\brAttr{Tipo}{Tipo}{Frase corta}{Frase que definirá el tipo de usuario que tienen las personas.}{Sí}
\end{cdtEntidad}
%---------------------------------------------------------
\begin{cdtEntidad}{ETS}{ETS}
	\brAttr{Id\_ETS}{Id\_ETS}{Id}{Número usado para identificar los diferentes ETS registrados.}{Sí}
	\brAttr{Id\_periodo}{Id\_periodo}{Id}
	{Número usado para identificar el periodo en el que se realiza el ETS.}{Sí}
	\brAttr{Turno}{Turno}{Caracter}
	{Letra usada identificar el turno en el que se aplica el ETS ('M' para matutino, 'V' para vespertino).}{Sí}
	\brAttr{Fecha}{Fecha}{Fecha}
	{Fecha y hora en la que se realizará el ETS.}{Sí}
	\brAttr{Cupo}{Cupo}{Número}
	{Número de personas permitidas a realizar el ETS.}{Sí}
	%- - - - - - - - - - - - - - - - - - - - - - - - - - - - -
	\cdtEntityRelSection
	\brRel{\brRelComposition}{periodo\_ETS}{Un \hyperlink{ETS}{ETS} se realiza un en \hyperlink{PETS}{periodo}}
	\brRel{\brRelComposition}{Alumno}{En un \hyperlink{ETS}{ETS} está \hyperlink{InscripcionETS}{inscrito} un \hyperlink{Alumno}{Alumno}}
	\brRel{\brRelComposition}{Docente}{Un \hyperlink{ETS}{ETS} es \hyperlink{Aplica}{aplicado} por un \hyperlink{Docente}{Docente}}
	\brRel{\brRelComposition}{Salon}{A un \hyperlink{ETS}{ETS} le \hyperlink{SalonETS}{corresponde} un \hyperlink{Salon}{Salón}}
	\brRel{\brRelComposition}{Unidad\_Aprendizaje}{Un \hyperlink{ETS}{ETS} \hyperlink{ETSUA}{es de} una \hyperlink{UA}{Unidad de Aprendizaje}.}
\end{cdtEntidad}
%---------------------------------------------------------
\begin{cdtEntidad}{PETS}{periodo\_ETS}
	\brAttr{Id\_periodo}{Id\_periodo}{Id}{Número usado para identificar el periodo en el que se realizarán los ETS registrados.}{Sí}
	\brAttr{Periodo}{Periodo}{Cadena de texto corta}
	{Periodo registrado en el que se realizarán los ETS.}{Sí}
	\brAttr{Tipo}{Tipo}{Caracter}
	{Letra usada identificar el tipo de los ETS que se aplicarán ('O' para ordinario, 'E' para especial).}{Sí}
	\brAttr{Fecha\_Inicio}{Fecha\_Inicio}{Fecha}
	{Fecha en la que iniciará el periodo de los ETS.}{Sí}
	\brAttr{Fecha\_Fin}{Fecha\_Fin}{Fecha}
	{Fecha en la que terminará el periodo de los ETS.}{Sí}
	%- - - - - - - - - - - - - - - - - - - - - - - - - - - - -
	\cdtEntityRelSection
	\brRel{\brRelComposition}{ETS}{En un \hyperlink{PETS}{periodo de ETS} se realizan los \hyperlink{ETS}{ETS}.}
\end{cdtEntidad}
%---------------------------------------------------------
\begin{cdtEntidad}{Aplica}{Aplica}
	\brAttr{Id\_ETS}{Id\_ETS}{Id}{Número usado para identificar los diferentes ETS registrados.}{Sí}
	\brAttr{DocenteRFC}{DocenteRFC}{Id}
	{RFC que identifica al Docente.}{Sí}
	\brAttr{Titular}{Titular}{Booleano}
	{Booleano que identificará si el profesor que aplicará el ETS es el titular o es un ayudante.}{Sí}
\end{cdtEntidad}
%---------------------------------------------------------
\begin{cdtEntidad}{InscripcionETS}{InscripcionETS}
	\brAttr{Boleta}{Boleta}{Id}
	{Código usado para identificar al alumnado de la institución}{Sí}
	\brAttr{Id\_ETS}{Id\_ETS}{Id}{Número usado para identificar los diferentes ETS registrados.}{Sí}
\end{cdtEntidad}
%---------------------------------------------------------
\begin{cdtEntidad}{SalonETS}{SalonETS}
	\brAttr{Num\_Salon}{Num\_Salon}{Id}
	{Número usado para identificar al salón en el que se aplicará un ETS.}{Sí}
	\brAttr{Id\_ETS}{Id\_ETS}{Id}{Número usado para identificar los diferentes ETS registrados.}{Sí}
\end{cdtEntidad}
%---------------------------------------------------------
\begin{cdtEntidad}{ETSUA}{ETS\_Unidad\_aprendizaje}
	\brAttr{Id\_ETS}{Id\_ETS}{Id}{Número usado para identificar los diferentes ETS registrados.}{Sí}
	\brAttr{Id\_UA}{Id\_UA}{Id}
	{Conjunto de caracteres usados para identificar a una unidad de aprendizaje.}{Sí}
\end{cdtEntidad}
%---------------------------------------------------------
\begin{cdtEntidad}{Salon}{Salon}
	\brAttr{Num\_Salon}{Num\_Salon}{Id}
	{Número usado para identificar el salón en el que se aplicará un ETS.}{Sí}
	\brAttr{Edificio}{Edificio}{Número}
	{Número usado para identificar el edificio en el que se realizará el ETS.}{Sí}
	\brAttr{Piso}{Piso}{Número}
	{Número usado para identificar el piso del edificio en el que se realizará el ETS.}{Sí}
	%- - - - - - - - - - - - - - - - - - - - - - - - - - - - -
	\cdtEntityRelSection
	\brRel{\brRelComposition}{ETS}{Un \hyperlink{Salon}{Salon} es \hyperlink{SalonETS}{ocupado} para llevar a cabo un \hyperlink{ETS}{ETS}.}
\end{cdtEntidad}
%---------------------------------------------------------
\begin{cdtEntidad}{UA}{Unidad\_aprendizaje}
	\brAttr{Num\_Salon}{Num\_Salon}{Id}
	{Número usado para identificar el salón en el que se aplicará un ETS.}{Sí}
	\brAttr{Edificio}{Edificio}{Número}
	{Número usado para identificar el edificio en el que se realizará el ETS.}{Sí}
	\brAttr{Piso}{Piso}{Número}
	{Número usado para identificar el piso del edificio en el que se realizará el ETS.}{Sí}
	%- - - - - - - - - - - - - - - - - - - - - - - - - - - - -
	\cdtEntityRelSection
	\brRel{\brRelComposition}{ETS}{A una \hyperlink{UA}{Unidad de aprendizaje} le \hyperlink{ETSUA}{corresponde} una serie de \hyperlink{ETS}{ETS}.}
\end{cdtEntidad}
%---------------------------------------------------------
\section{Modelado de Reglas de negocio}

\input{3-1-reglas}

